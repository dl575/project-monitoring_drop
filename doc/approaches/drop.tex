\documentclass[11pt, letterpaper]{article}
	\title{Dropping Algorithms for Selective Monitoring}
	\author{Daniel Lo \{dl575\}}
	
	% Packages for math formatting
	\usepackage{amsfonts}
	\usepackage{amsmath}
	\usepackage{amsthm}

	% for letter paper
	\usepackage[letterpaper]{geometry}
	% 1 inch margins
	\usepackage{fullpage}    
	% allow enumerate numberings to be specified
	\usepackage{enumerate}
	% Insert images
	\usepackage{graphicx}
	% For syntax highlighting
	\usepackage{listings}
  % Graphviz
  \usepackage{graphviz}
  % Algorithms
  \usepackage{algorithm}
  \usepackage{algorithmic}
		
	% Header
	\usepackage{fancyhdr}
	\pagestyle{fancy}
	\headheight 30pt
	\rhead{}
	\lhead{Dropping Algorithms for Selective Monitoring \\
  Daniel Lo \{dl575@cornell.edu\}}
  \headsep 0.1in
	
	% Define a problem "theorem" heading
	\newtheorem{problem}{Problem}
	% Proposition theorem
	\newtheorem{proposition}{Proposition}
	% Scientific form
	\providecommand{\e}[1]{\ensuremath{\cdot 10^{#1}}}
	% Matrix - use mathbf font
	\providecommand{\m}[1]{\mathbf{#1}}
	% Degree symbol
	\providecommand{\degrees}{^{\circ}}
	% Insert figure
	\providecommand{\fig}[1]{
		\noindent
		\begin{center}
			\includegraphics[height=3.5in]{#1}
		\end{center}
	}
	\providecommand{\dualfig}[2]{
		\noindent
		\begin{center}
		\includegraphics[width=3.2in]{#1} \hspace{-0.3in}
		\includegraphics[width=3.2in]{#2}
		\end{center}
	}
	\providecommand{\tripfig}[3]{
		\noindent
		\begin{center}
		\hspace{-0.3in}
		\includegraphics[width=2.3in]{#1} \hspace{-0.3in}
		\includegraphics[width=2.3in]{#2} \hspace{-0.3in}
		\includegraphics[width=2.3in]{#3} \hspace{-0.3in}
		\end{center}
	}
	% begin/end align*
	\providecommand{\eq}[1]{
		\begin{align*}
		#1
		\end{align*}
	}
	% Kernel/Image
	\providecommand{\im}[1]{
		\text{Im}(#1)
	}
	\renewcommand{\ker}[1]{
		\text{Ker}(#1)
	}
	% Trace
	\providecommand{\tr}[1]{
		\text{Tr}\left(#1\right)
	}
	% Such that
	\providecommand{\st}[0]{
		%\text{ s.t. }
		\ni
	}
	% Real numbers
	\providecommand{\reals}[0]{
		\mathbb{R}
	}
	% Use overline which is longer instead of bar
	\renewcommand{\bar}[1]{
		\overline{#1}
	}
	% Create divided matrix representing state-space system
	\providecommand{\statespace}[4]{
		\begin{bmatrix}
		\begin{array}{c|c}
			#1 & #2 \\
			\hline
			#3 & #4
		\end{array}
		\end{bmatrix}
	}
	% Include matlab code
	\providecommand{\matlab}[1]{
	  \lstinputlisting[language=matlab,
		showstringspaces=false,
		basicstyle=\footnotesize]
		{#1}
	}
  % Generate graph using dot/graphviz
  \providecommand{\dotgraph}[2]{
    \vspace{-0.6in}
    \begin{center}
    \digraph[scale=0.75]{#1}{#2}
    \end{center}
    \vspace{-0.6in}
  }
		
	
	% Insert a blank line between paragraphs
	\setlength{\parskip}{\baselineskip}
\begin{document}

%\maketitle

\section{Introduction}
In this document, we discuss possible coverage objectives and several dropping
algorithms that target different objectives and work well in different
contexts.

\section{Coverage}

There are several ways to define monitoring coverage. In all cases, we consider
only check operations (and not update operations) as contributing to coverage.

First, coverage can either be static coverage or dynamic coverage. Static
coverage refers to the portion of static code instructions that are monitored
at least once. Dynamic coverage refers to the portion of dynamic check event
instances that are monitored. For example, consider the following piece of code.

\begin{center}
\begin{tabular}{|c|}
\hline
\\
\begin{lstlisting}[language=c]
for (i = 0; i < N; i++) {
  input = get_user_input();
  memcpy(&input, &output, MAX_INPUT);
}
\end{lstlisting}
\\\\ \hline
\end{tabular}
\end{center}

Suppose array bounds check is used to check for buffer overflows on the {\tt memcpy}.
Dynamic coverage is calculated as the number of iterations that are checked
divided by $N$. On the other hand, if {\tt memcpy} is checked on at least one
iteration of the loop, then static coverage is 100\%. In this sense, dynamic
coverage is a better metric for detecting \emph{attacks} such as for BC and
DIFT. On the other hand, static coverage gives a measure of code coverage for
finding \emph{code bugs} such as for UMC.

The other attribute we can consider is whether coverage is measured for a
single run or for multiple runs. Optimizing for multiple-run coverage is good
for \emph{detection}. The goal is to detect all errors over multiple runs
and/or users. However, the chance that an attack is detected for a given user
is poor. Instead, optimizing for single-run coverage is good for
\emph{protection}. Although certain checks may never be done by any user, each
user will be doing a maximum number of checks, reducing the number of dropped
checks or ``unprotected portions''.

The following chart shows the spectrum of coverage targets.

\begin{center}
\begin{tabular}{c|c|c|}
 & \textbf{Static Coverage} & \textbf{Dynamic Coverage} \\ \hline
\textbf{Single run} & Developer Tool & Attack Prevention \\ \hline
\textbf{Multiple run} & Bug Detection & Attack Detection \\ \hline
\end{tabular}
\end{center}

Providing single-run, static coverage is useful for a developer or debugging
tool where the user wants to run the code with monitoring once to check that
written code does not have obvious bugs. Single-run, dynamic coverage works
well for preventing attacks by minimizing the dropped dynamic checks.
Multiple-run, static coverage is good for detecting code bugs since over
multiple users static instructions should be checked. Multiple-run, dynamic
coverage is good for detecting attacks since over multiple users a large number
of dynamic check instances should be covered, decreasing the chance of an
attack point being undetected.

\subsection{Related Work}

Our previous real-time work \footnote{Lo et al. Slack-Aware Opportunistic
Monitoring for Real-Time Systems. RTAS 2014.} optimizes for dynamic coverage on
a single run. This is done while providing a strict guarantee of overhead.

Huang et al. \footnote{Huang et al. Software Monitoring with Controllable
Overhead. STTT 2012.} enable or disable monitoring interrupts with controllers
designed using control theory concepts in order to match an overhead target.
This targets dynamic coverage on a single run.

Greathouse et al. \footnote{Greathouse et al. Highly Scalable Distributed
Dataflow Analysis. CGO 2011.} performs dynamic dataflow analysis with an
overhead target. If a performance threshold is crossed, then dataflows are
randomly discarded (clear taint). This randomness provides good dynamic
coverage over multiple runs assuming flows are short. For long chains and high
overheads, it's unlikely that all operations will be performed unless the
random probability is set low enough.

QVM \footnote{Arnold et al. QVM: An Efficient Runtime for Detecting Defects in
Deployed Systems. OOPSLA 2008.} modified the JVM to support monitoring with adjustable
overheads by adjusting the rate of monitoring for each object allocation site.
Since the rate is tied to object allocation sites, static coverage is maximized
with dynamic coverage as a secondary objective. Thus, QVM targets static
coverage on a single run.

Testudo \footnote{Greathouse et al. Testudo: Heavyweight Security Analysis via
Statistical Sampling. MICRO 2008.} monitors only one DIFT flow at a time in order to reduce
overheads. The choice of flow is done randomly in order to provide good
coverage over multiple users. Thus, Testudo provides good dynamic coverage over
multiple runs with no explicit overhead target.

The following chart attempts to classify these approaches in terms of coverage
target.
\begin{center}
\begin{tabular}{c|c|c|}
 & \textbf{Static Coverage} & \textbf{Dynamic Coverage} \\ \hline
\textbf{Single run} & QVM & Lo et al., Huang et al. \\ \hline
\textbf{Multiple run} & & Testudo, Greathouse et al. \\ \hline
\end{tabular}
\end{center}

\section{Challenges}
\subsection{Chain of Randomness}

\dotgraph{random_chain}{
  rankdir=LR; a->b; b->c; c->d; d->e;
}
Introducing randomness can be used to improve coverage over multiple runs.
However, if there are multiple dependent nodes (as in the above figure) and
each is randomly dropped, then it is unlikely that the final node (i.e, node e)
will be monitored. For example, if each node has a 20\% probability of being
dropped, then the probability that monitoring is done for node e is only
$1-0.80^5 = 33\%$.

\subsection{Overhead Target vs. Multiple-Run Coverage}
There is an inherent tension between meeting an overhead target and achieving
good coverage over multiple runs. Optimizing for good coverage over multiple
runs implies that randomness should be used. However, only using randomness is
not necessarily able to meet an overhead target. On the other hand, if dropping
decisions are made based on an overhead target, then bursty sections of the
program will see lower monitoring for all runs/users.

This can be seen in previous work. Techniques that target an overhead do not
use randomness and do not work well for multiple-run coverage. On the other
hand, Testudo achieves good multiple-run coverage using randomness, but does
not take into account an overhead target at all. The work by Greathouse et al.
randomly drops flows when the overhead is over the target. This attempts to
provide an overhead target and multiple-run coverage. However, this works
poorly if overheads are high and flows are large due to the ``Chain of
Randomness'' challenge.

\subsection{Early vs. Late Usage of Slack}
One source of bias in monitoring for an overhead target is that earlier
monitoring operations may use up budget for later events. For example, suppose
there is a burst of monitoring operations in the middle of the program and
another burst at the end of the program. When the group of monitoring
operations in the middle of the program is seen, it is likely to be performed
because overheads are low. However, if these cause larger overheads, then it is
unlikely that the monitoring operations at the end of the program will be
performed. In order for the monitoring operations at the end of the program to
be performed, the monitoring operations in the middle of the program must be
dropped. This is difficult to handle because at the middle of the program
execution, it may not be known whether there are operations later for which
overhead/slack should be saved. This is somewhat similar to the ``Chain of
Randomness'' challenge but dealing with temporal location of events.

\section{Approaches}
This section discusses several dropping decision approaches.

\subsection{Source Dropping}
\label{sec:source}
\textbf{When: } Dropping decision is made at sources (set tag operations). \\
\textbf{How: } Randomly decide whether to drop or not. Adjust probability based
on overhead target and actual overhead. \\
\textbf{Optimizes for: } Multiple-run dynamic coverage. \\
\textbf{Pros/Cons: }
\begin{itemize}
  \item[+] No wasted work
  \item[+] Works well for many small flows
  \item[--] Works poorly for a few large flows (hard to match overhead target)
  \item[--] Does not work if a single flow exceeds overhead target
\end{itemize}

\subsection{Static Dropping}
\label{sec:static}
\textbf{When: } Use static analysis to find dependence graph and determine
which checks and nodes to perform. \\
\textbf{How: } Choose minimal checks dropped such that number of nodes dropped
is estimated to meet overhead target. \\
\textbf{Optimizes for: } Any type of coverage depending on how selection is done. \\
\textbf{Pros/Cons: }
\begin{itemize}
  \item[+] No wasted work
  \item[--] Difficult to match overhead (need to use estimate based on nodes dropped)
\end{itemize}

\subsection{Optimal-Point Dropping}
\label{sec:opt}
\textbf{When: } Use dependence graph to determine ``optimal'' dropping points
(points which lead to no wasted work). Perform dropping decision at these
points. \\
\textbf{How: } Static analysis calculates number of checks and nodes dropped at
the optimal dropping points. This provides the cost and benefit of dropping the
node. The cost:benefit ratio is compared to a threshold that is dynamically
adjusted in an attempt to meet the overhead target. In addition, there is a
random probability for each optimal dropping point to be dropped, independent
of the threshold decision. \\
\textbf{Optimizes for: Dynamic coverage over multiple runs.} \\
\textbf{Pros/Cons: }
\begin{itemize}
  \item[+] Minimal wasted work (only occurs for merges and only half is dropped)
  \item[+] Can better match overheads for large flows than Approach~\ref{sec:source}
  \item[+] Randomness works well for multiple-run coverage if number of dropping points in a chain are low
  \item[--] Randomness works poorly for multiple-run coverage if many dropping points in a chain
\end{itemize}

\end{document}
